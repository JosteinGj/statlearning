\documentclass[]{article}
\usepackage{lmodern}
\usepackage{amssymb,amsmath}
\usepackage{ifxetex,ifluatex}
\usepackage{fixltx2e} % provides \textsubscript
\ifnum 0\ifxetex 1\fi\ifluatex 1\fi=0 % if pdftex
  \usepackage[T1]{fontenc}
  \usepackage[utf8]{inputenc}
\else % if luatex or xelatex
  \ifxetex
    \usepackage{mathspec}
  \else
    \usepackage{fontspec}
  \fi
  \defaultfontfeatures{Ligatures=TeX,Scale=MatchLowercase}
\fi
% use upquote if available, for straight quotes in verbatim environments
\IfFileExists{upquote.sty}{\usepackage{upquote}}{}
% use microtype if available
\IfFileExists{microtype.sty}{%
\usepackage{microtype}
\UseMicrotypeSet[protrusion]{basicmath} % disable protrusion for tt fonts
}{}
\usepackage[margin=1in]{geometry}
\usepackage{hyperref}
\hypersetup{unicode=true,
            pdftitle={Module 3: Recommended Exercises},
            pdfauthor={Stefanie Muff, Department of Mathematical Sciences, NTNU},
            pdfborder={0 0 0},
            breaklinks=true}
\urlstyle{same}  % don't use monospace font for urls
\usepackage{color}
\usepackage{fancyvrb}
\newcommand{\VerbBar}{|}
\newcommand{\VERB}{\Verb[commandchars=\\\{\}]}
\DefineVerbatimEnvironment{Highlighting}{Verbatim}{commandchars=\\\{\}}
% Add ',fontsize=\small' for more characters per line
\usepackage{framed}
\definecolor{shadecolor}{RGB}{248,248,248}
\newenvironment{Shaded}{\begin{snugshade}}{\end{snugshade}}
\newcommand{\KeywordTok}[1]{\textcolor[rgb]{0.13,0.29,0.53}{\textbf{#1}}}
\newcommand{\DataTypeTok}[1]{\textcolor[rgb]{0.13,0.29,0.53}{#1}}
\newcommand{\DecValTok}[1]{\textcolor[rgb]{0.00,0.00,0.81}{#1}}
\newcommand{\BaseNTok}[1]{\textcolor[rgb]{0.00,0.00,0.81}{#1}}
\newcommand{\FloatTok}[1]{\textcolor[rgb]{0.00,0.00,0.81}{#1}}
\newcommand{\ConstantTok}[1]{\textcolor[rgb]{0.00,0.00,0.00}{#1}}
\newcommand{\CharTok}[1]{\textcolor[rgb]{0.31,0.60,0.02}{#1}}
\newcommand{\SpecialCharTok}[1]{\textcolor[rgb]{0.00,0.00,0.00}{#1}}
\newcommand{\StringTok}[1]{\textcolor[rgb]{0.31,0.60,0.02}{#1}}
\newcommand{\VerbatimStringTok}[1]{\textcolor[rgb]{0.31,0.60,0.02}{#1}}
\newcommand{\SpecialStringTok}[1]{\textcolor[rgb]{0.31,0.60,0.02}{#1}}
\newcommand{\ImportTok}[1]{#1}
\newcommand{\CommentTok}[1]{\textcolor[rgb]{0.56,0.35,0.01}{\textit{#1}}}
\newcommand{\DocumentationTok}[1]{\textcolor[rgb]{0.56,0.35,0.01}{\textbf{\textit{#1}}}}
\newcommand{\AnnotationTok}[1]{\textcolor[rgb]{0.56,0.35,0.01}{\textbf{\textit{#1}}}}
\newcommand{\CommentVarTok}[1]{\textcolor[rgb]{0.56,0.35,0.01}{\textbf{\textit{#1}}}}
\newcommand{\OtherTok}[1]{\textcolor[rgb]{0.56,0.35,0.01}{#1}}
\newcommand{\FunctionTok}[1]{\textcolor[rgb]{0.00,0.00,0.00}{#1}}
\newcommand{\VariableTok}[1]{\textcolor[rgb]{0.00,0.00,0.00}{#1}}
\newcommand{\ControlFlowTok}[1]{\textcolor[rgb]{0.13,0.29,0.53}{\textbf{#1}}}
\newcommand{\OperatorTok}[1]{\textcolor[rgb]{0.81,0.36,0.00}{\textbf{#1}}}
\newcommand{\BuiltInTok}[1]{#1}
\newcommand{\ExtensionTok}[1]{#1}
\newcommand{\PreprocessorTok}[1]{\textcolor[rgb]{0.56,0.35,0.01}{\textit{#1}}}
\newcommand{\AttributeTok}[1]{\textcolor[rgb]{0.77,0.63,0.00}{#1}}
\newcommand{\RegionMarkerTok}[1]{#1}
\newcommand{\InformationTok}[1]{\textcolor[rgb]{0.56,0.35,0.01}{\textbf{\textit{#1}}}}
\newcommand{\WarningTok}[1]{\textcolor[rgb]{0.56,0.35,0.01}{\textbf{\textit{#1}}}}
\newcommand{\AlertTok}[1]{\textcolor[rgb]{0.94,0.16,0.16}{#1}}
\newcommand{\ErrorTok}[1]{\textcolor[rgb]{0.64,0.00,0.00}{\textbf{#1}}}
\newcommand{\NormalTok}[1]{#1}
\usepackage{graphicx,grffile}
\makeatletter
\def\maxwidth{\ifdim\Gin@nat@width>\linewidth\linewidth\else\Gin@nat@width\fi}
\def\maxheight{\ifdim\Gin@nat@height>\textheight\textheight\else\Gin@nat@height\fi}
\makeatother
% Scale images if necessary, so that they will not overflow the page
% margins by default, and it is still possible to overwrite the defaults
% using explicit options in \includegraphics[width, height, ...]{}
\setkeys{Gin}{width=\maxwidth,height=\maxheight,keepaspectratio}
\IfFileExists{parskip.sty}{%
\usepackage{parskip}
}{% else
\setlength{\parindent}{0pt}
\setlength{\parskip}{6pt plus 2pt minus 1pt}
}
\setlength{\emergencystretch}{3em}  % prevent overfull lines
\providecommand{\tightlist}{%
  \setlength{\itemsep}{0pt}\setlength{\parskip}{0pt}}
\setcounter{secnumdepth}{0}
% Redefines (sub)paragraphs to behave more like sections
\ifx\paragraph\undefined\else
\let\oldparagraph\paragraph
\renewcommand{\paragraph}[1]{\oldparagraph{#1}\mbox{}}
\fi
\ifx\subparagraph\undefined\else
\let\oldsubparagraph\subparagraph
\renewcommand{\subparagraph}[1]{\oldsubparagraph{#1}\mbox{}}
\fi

%%% Use protect on footnotes to avoid problems with footnotes in titles
\let\rmarkdownfootnote\footnote%
\def\footnote{\protect\rmarkdownfootnote}

%%% Change title format to be more compact
\usepackage{titling}

% Create subtitle command for use in maketitle
\providecommand{\subtitle}[1]{
  \posttitle{
    \begin{center}\large#1\end{center}
    }
}

\setlength{\droptitle}{-2em}

  \title{Module 3: Recommended Exercises}
    \pretitle{\vspace{\droptitle}\centering\huge}
  \posttitle{\par}
  \subtitle{TMA4268 Statistical Learning V2020}
  \author{Stefanie Muff, Department of Mathematical Sciences, NTNU}
    \preauthor{\centering\large\emph}
  \postauthor{\par}
      \predate{\centering\large\emph}
  \postdate{\par}
    \date{January xx, 2020}


\begin{document}
\maketitle

{
\setcounter{tocdepth}{2}
\tableofcontents
}
\subsection{Problem 1: Compulsory exercise in
2018}\label{problem-1-compulsory-exercise-in-2018}

There will be a very similar regression problem in the compulsory
exercise 1 in 2019!

The Framingham Heart Study is a study of the etiology (i.e.~underlying
causes) of cardiovascular disease, with participants from the community
of Framingham in Massachusetts, USA. For more more information about the
Framingham Heart Study visit
\url{https://www.framinghamheartstudy.org/}. The dataset used in here is
subset of a teaching version of the Framingham data, used with
permission from the Framingham Heart Study.

We will focus on modelling systolic blood pressure using data from n =
2600 persons. For each person in the data set we have measurements of
the seven variables

\begin{itemize}
\tightlist
\item
  \texttt{SYSBP} systolic blood pressure,
\item
  \texttt{SEX} 1=male, 2=female,
\item
  \texttt{AGE} age in years at examination,
\item
  \texttt{CURSMOKE} current cigarette smoking at examination: 0=not
  current smoker, 1= current smoker,
\item
  \texttt{BMI} body mass index,
\item
  \texttt{TOTCHOL} serum total cholesterol, and
\item
  \texttt{BPMEDS} use of anti-hypertensive medication at examination:
  0=not currently using, 1=currently using.
\end{itemize}

A multiple normal linear regression model was fitted to the data set
with \texttt{-1/sqrt(SYSBP)} as response and all the other variables as
covariates.

\begin{Shaded}
\begin{Highlighting}[]
\KeywordTok{library}\NormalTok{(ggplot2)}
\NormalTok{data =}\StringTok{ }\KeywordTok{read.table}\NormalTok{(}\StringTok{"https://www.math.ntnu.no/emner/TMA4268/2018v/data/SYSBPreg3uid.txt"}\NormalTok{)}
\KeywordTok{dim}\NormalTok{(data)}
\KeywordTok{colnames}\NormalTok{(data)}
\NormalTok{modelA =}\StringTok{ }\KeywordTok{lm}\NormalTok{(}\OperatorTok{-}\DecValTok{1}\OperatorTok{/}\KeywordTok{sqrt}\NormalTok{(SYSBP) }\OperatorTok{~}\StringTok{ }\NormalTok{., }\DataTypeTok{data =}\NormalTok{ data)}
\KeywordTok{summary}\NormalTok{(modelA)}
\end{Highlighting}
\end{Shaded}

\subsection{a) Understanding model
output}\label{a-understanding-model-output}

We name the model fitted above \texttt{modelA}.

\begin{itemize}
\tightlist
\item
  Write down the equation for the fitted \texttt{modelA}.
\item
  Explain (with words and formula) what the following in the
  \texttt{summary}-output means.
\item
  \texttt{Estimate} - in particular interpretation of \texttt{Intercept}
\item
  \texttt{Std.Error}
\item
  \texttt{t\ value}
\item
  \texttt{Pr(\textgreater{}\textbar{}t\textbar{})}
\item
  \texttt{Residual\ standard\ error}
\item
  \texttt{F-statistic}
\end{itemize}

\subsection{b) Model fit}\label{b-model-fit}

\begin{itemize}
\tightlist
\item
  What is the proportion of variability explained by the fitted
  \texttt{modelA}? Comment.
\item
  Use diagnostic plots of ``fitted values vs.~standardized residuals''"
  and ``QQ-plot of standardized residuals'' (see code below) to assess
  the model fit.
\item
  Now fit a model, call this \texttt{modelB}, with \texttt{SYSBP} as
  response, and the same covariates as for \texttt{modelA}. Would you
  prefer to use \texttt{modelA} or \texttt{modelB} when the aim is to
  make inference about the systolic blood pressure?
\end{itemize}

\begin{Shaded}
\begin{Highlighting}[]
\CommentTok{# residuls vs fitted}
\KeywordTok{ggplot}\NormalTok{(modelA, }\KeywordTok{aes}\NormalTok{(.fitted, .resid)) }\OperatorTok{+}\StringTok{ }\KeywordTok{geom_point}\NormalTok{(}\DataTypeTok{pch =} \DecValTok{21}\NormalTok{) }\OperatorTok{+}\StringTok{ }\KeywordTok{geom_hline}\NormalTok{(}\DataTypeTok{yintercept =} \DecValTok{0}\NormalTok{, }
    \DataTypeTok{linetype =} \StringTok{"dashed"}\NormalTok{) }\OperatorTok{+}\StringTok{ }\KeywordTok{geom_smooth}\NormalTok{(}\DataTypeTok{se =} \OtherTok{FALSE}\NormalTok{, }\DataTypeTok{col =} \StringTok{"red"}\NormalTok{, }\DataTypeTok{size =} \FloatTok{0.5}\NormalTok{, }
    \DataTypeTok{method =} \StringTok{"loess"}\NormalTok{) }\OperatorTok{+}\StringTok{ }\KeywordTok{labs}\NormalTok{(}\DataTypeTok{x =} \StringTok{"Fitted values"}\NormalTok{, }\DataTypeTok{y =} \StringTok{"Residuals"}\NormalTok{, }\DataTypeTok{title =} \StringTok{"Fitted values vs. residuals"}\NormalTok{, }
    \DataTypeTok{subtitle =} \KeywordTok{deparse}\NormalTok{(modelA}\OperatorTok{$}\NormalTok{call))}

\CommentTok{# qq-plot of residuals}
\KeywordTok{ggplot}\NormalTok{(modelA, }\KeywordTok{aes}\NormalTok{(}\DataTypeTok{sample =}\NormalTok{ .stdresid)) }\OperatorTok{+}\StringTok{ }\KeywordTok{stat_qq}\NormalTok{(}\DataTypeTok{pch =} \DecValTok{19}\NormalTok{) }\OperatorTok{+}\StringTok{ }\KeywordTok{geom_abline}\NormalTok{(}\DataTypeTok{intercept =} \DecValTok{0}\NormalTok{, }
    \DataTypeTok{slope =} \DecValTok{1}\NormalTok{, }\DataTypeTok{linetype =} \StringTok{"dotted"}\NormalTok{) }\OperatorTok{+}\StringTok{ }\KeywordTok{labs}\NormalTok{(}\DataTypeTok{x =} \StringTok{"Theoretical quantiles"}\NormalTok{, }
    \DataTypeTok{y =} \StringTok{"Standardized residuals"}\NormalTok{, }\DataTypeTok{title =} \StringTok{"Normal Q-Q"}\NormalTok{, }\DataTypeTok{subtitle =} \KeywordTok{deparse}\NormalTok{(modelA}\OperatorTok{$}\NormalTok{call))}
\end{Highlighting}
\end{Shaded}

\subsection{c) Confidence interval and hypothesis
test}\label{c-confidence-interval-and-hypothesis-test}

We use \texttt{modelA} and focus on addressing the association between
BMI and the response.

\begin{itemize}
\tightlist
\item
  What is the estimate \(\hat{\beta}_{\text{BMI}}\) (numerically)?
\item
  Explain how to interpret the estimated coefficient
  \(\hat{\beta}_{\text{BMI}}\).
\item
  Construct a 99\% confidence interval for \(\beta_{\text{BMI}}\) (write
  out the formula and calculate the interval numerically). Explain what
  this interval tells you.
\item
  From this confidence interval, is it possible for you know anything
  about the value of the \(p\)-value for the test
  \(H_0: \beta_{\text{BMI}}=0\) vs. \(H_1:\beta_{\text{BMI}} \neq 0\)?
  Explain.
\end{itemize}

\subsection{d) Prediction}\label{d-prediction}

Consider a 56 year old man who is smoking. He is 1.75 meters tall and
his weight is 89 kilograms. His serum total cholesterol is 200 mg/dl and
he is not using anti-hypertensive medication.

\begin{Shaded}
\begin{Highlighting}[]
\KeywordTok{names}\NormalTok{(data)}
\NormalTok{new =}\StringTok{ }\KeywordTok{data.frame}\NormalTok{(}\DataTypeTok{SEX =} \DecValTok{1}\NormalTok{, }\DataTypeTok{AGE =} \DecValTok{56}\NormalTok{, }\DataTypeTok{CURSMOKE =} \DecValTok{1}\NormalTok{, }\DataTypeTok{BMI =} \DecValTok{89}\OperatorTok{/}\FloatTok{1.75}\OperatorTok{^}\DecValTok{2}\NormalTok{, }\DataTypeTok{TOTCHOL =} \DecValTok{200}\NormalTok{, }
    \DataTypeTok{BPMEDS =} \DecValTok{0}\NormalTok{)}
\end{Highlighting}
\end{Shaded}

\begin{itemize}
\tightlist
\item
  What is your best guess for his \texttt{-1/sqrt(SYSBP)}? To get a best
  guess for his \texttt{SYSBP} you may take the inverse function of
  \texttt{-1/sqrt}.
\end{itemize}

(Comment: Is that allowed - to only do the inverse? Yes, that could be
the result of a first order Taylor expansion approximation.)

\begin{itemize}
\tightlist
\item
  Construct a 90\% prediction interval for his systolic blood pressure
  \texttt{SYSBP}. Comment. Hint: first contruct values on the scale of
  the response \texttt{-1/sqrt(SYSBP)} and then transform the upper and
  lower limits of the prediction interval.
\item
  Do you find this prediction interval useful? Comment.
\end{itemize}

\begin{center}\rule{0.5\linewidth}{\linethickness}\end{center}

\subsection{Problem 2: Theoretical
questions}\label{problem-2-theoretical-questions}

\subsubsection{a)}\label{a}

A core finding is \(\hat{\boldsymbol\beta}\).
\[ \hat{\boldsymbol\beta}=({\bf X}^T{\bf X})^{-1} {\bf X}^T {\bf Y}\]
with
\(\hat{\boldsymbol\beta}\sim N_{p}(\boldsymbol\beta,\sigma^2({\bf X}^T{\bf X})^{-1})\).

\begin{itemize}
\tightlist
\item
  Show that \(\hat{\boldsymbol\beta}\) has this distribution with the
  given mean and covariance matrix.
\item
  What do you need to assume to get to this result?
\item
  What does this imply for the distribution of the \(j\)th element of
  \(\hat{\beta}\)?
\item
  In particular, how can we calculate the variance of \(\hat{\beta}_j\)?
\end{itemize}

\subsubsection{b)}\label{b}

What is the interpretation of a 95\% confidence interval? Hint: repeat
experiment (on \(Y\)), on average how many CIs cover the true
\(\beta_j\)?

\subsubsection{c)}\label{c}

What is the interpretation of a 95\% prediction interval? Hint: repeat
experiment (on \(Y\)) for a given \({\bf x}_0\).

\subsubsection{d)}\label{d}

Construct a 95\% CI for \({\bf x}_0^T \beta\). Explain what is the
connections between a CI for \(\beta_j\), a CI for \({\bf x}_0^T \beta\)
and a PI for \(Y\) at \({\bf x}_0\).

\subsubsection{e)}\label{e}

Explain the difference between \emph{error} and \emph{residual}. What
are the properties of the raw residuals? Why don't we want to use the
raw residuals for model check? What is our solution to this?

\subsubsection{f)}\label{f}

Consider a multiple linear regression model \(A\) and a submodel \(B\)
(all parameters in \(B\) are in \(A\) also). We say that \(B\) is nested
within \(A\). Assume that regression parameters are estimated using
least squares. Why is then the following true: RSS for model \(A\) will
always be smaller or equal to RSS for model \(B\). And thus, \(R^2\) for
model \(A\) can never be worse than \(R^2\) for model B. (See also
Problem 3d below.)

\begin{center}\rule{0.5\linewidth}{\linethickness}\end{center}

\subsection{Problem 3: Munich Rent
index}\label{problem-3-munich-rent-index}

\subsubsection{a)}\label{a-1}

Fit the regression model with first \texttt{rent} and then
\texttt{rentsqm} as response and following covariates: \texttt{area},
\texttt{location} (dummy variable coding using location2 and location3,
just write \texttt{as.factor(location)}), \texttt{bath},
\texttt{kitchen} and \texttt{cheating} (central heating).

Look at diagnostic plots for the two fits. Which response do you prefer?

Consentrate on the response-model you choose for the rest of the tasks.

\subsubsection{b)}\label{b-1}

Explain what the parameter estimates mean in practice. In particular,
what is the interpretation of the intercept?

\subsubsection{c)}\label{c-1}

Go through the summary printout and explain all parts.

\subsubsection{d)}\label{d-1}

Now we add random noise as a covariance, but simulating the IQ of the
landlord of each appartment. Observe that \(R^2\) increases (or stays
unchanged) and RSS decreases (or stays the same) if we add IQ as
covariate, but \(R^2_{\text{adj}}\) decreases. What does this tell you
about model selection and overfitting?

For the code - what is the connection between \texttt{sigma} and RSS?

\begin{Shaded}
\begin{Highlighting}[]
\KeywordTok{library}\NormalTok{(gamlss.data)}
\NormalTok{orgfit =}\StringTok{ }\KeywordTok{lm}\NormalTok{(rent }\OperatorTok{~}\StringTok{ }\NormalTok{area }\OperatorTok{+}\StringTok{ }\KeywordTok{as.factor}\NormalTok{(location) }\OperatorTok{+}\StringTok{ }\NormalTok{bath }\OperatorTok{+}\StringTok{ }\NormalTok{kitchen }\OperatorTok{+}\StringTok{ }\NormalTok{cheating, }
    \DataTypeTok{data =}\NormalTok{ rent99)}
\KeywordTok{summary}\NormalTok{(orgfit)}
\KeywordTok{set.seed}\NormalTok{(}\DecValTok{1}\NormalTok{)  }\CommentTok{#to be able to reproduce results}
\NormalTok{n =}\StringTok{ }\KeywordTok{dim}\NormalTok{(rent99)[}\DecValTok{1}\NormalTok{]}
\NormalTok{IQ =}\StringTok{ }\KeywordTok{rnorm}\NormalTok{(n, }\DecValTok{100}\NormalTok{, }\DecValTok{16}\NormalTok{)}
\NormalTok{fitIQ =}\StringTok{ }\KeywordTok{lm}\NormalTok{(rent }\OperatorTok{~}\StringTok{ }\NormalTok{area }\OperatorTok{+}\StringTok{ }\KeywordTok{as.factor}\NormalTok{(location) }\OperatorTok{+}\StringTok{ }\NormalTok{bath }\OperatorTok{+}\StringTok{ }\NormalTok{kitchen }\OperatorTok{+}\StringTok{ }\NormalTok{cheating }\OperatorTok{+}\StringTok{ }
\StringTok{    }\NormalTok{IQ, }\DataTypeTok{data =}\NormalTok{ rent99)}
\KeywordTok{summary}\NormalTok{(fitIQ)}

\KeywordTok{summary}\NormalTok{(orgfit)}\OperatorTok{$}\NormalTok{sigma}
\KeywordTok{summary}\NormalTok{(fitIQ)}\OperatorTok{$}\NormalTok{sigma}

\KeywordTok{summary}\NormalTok{(orgfit)}\OperatorTok{$}\NormalTok{r.squared}
\KeywordTok{summary}\NormalTok{(fitIQ)}\OperatorTok{$}\NormalTok{r.squared}
\KeywordTok{summary}\NormalTok{(orgfit)}\OperatorTok{$}\NormalTok{adj.r.squared}
\KeywordTok{summary}\NormalTok{(fitIQ)}\OperatorTok{$}\NormalTok{adj.r.squared}
\end{Highlighting}
\end{Shaded}

\begin{verbatim}
## 
## Call:
## lm(formula = rent ~ area + as.factor(location) + bath + kitchen + 
##     cheating, data = rent99)
## 
## Residuals:
##     Min      1Q  Median      3Q     Max 
## -633.41  -89.17   -6.26   82.96 1000.76 
## 
## Coefficients:
##                      Estimate Std. Error t value Pr(>|t|)    
## (Intercept)          -21.9733    11.6549  -1.885   0.0595 .  
## area                   4.5788     0.1143  40.055  < 2e-16 ***
## as.factor(location)2  39.2602     5.4471   7.208 7.14e-13 ***
## as.factor(location)3 126.0575    16.8747   7.470 1.04e-13 ***
## bath1                 74.0538    11.2087   6.607 4.61e-11 ***
## kitchen1             120.4349    13.0192   9.251  < 2e-16 ***
## cheating1            161.4138     8.6632  18.632  < 2e-16 ***
## ---
## Signif. codes:  0 '***' 0.001 '**' 0.01 '*' 0.05 '.' 0.1 ' ' 1
## 
## Residual standard error: 145.2 on 3075 degrees of freedom
## Multiple R-squared:  0.4504, Adjusted R-squared:  0.4494 
## F-statistic:   420 on 6 and 3075 DF,  p-value: < 2.2e-16
## 
## 
## Call:
## lm(formula = rent ~ area + as.factor(location) + bath + kitchen + 
##     cheating + IQ, data = rent99)
## 
## Residuals:
##     Min      1Q  Median      3Q     Max 
## -630.95  -89.50   -6.12   82.62  995.76 
## 
## Coefficients:
##                      Estimate Std. Error t value Pr(>|t|)    
## (Intercept)          -41.3879    19.5957  -2.112   0.0348 *  
## area                   4.5785     0.1143  40.056  < 2e-16 ***
## as.factor(location)2  39.2830     5.4467   7.212 6.90e-13 ***
## as.factor(location)3 126.3356    16.8748   7.487 9.18e-14 ***
## bath1                 74.1979    11.2084   6.620 4.23e-11 ***
## kitchen1             120.0756    13.0214   9.221  < 2e-16 ***
## cheating1            161.4450     8.6625  18.637  < 2e-16 ***
## IQ                     0.1940     0.1574   1.232   0.2179    
## ---
## Signif. codes:  0 '***' 0.001 '**' 0.01 '*' 0.05 '.' 0.1 ' ' 1
## 
## Residual standard error: 145.2 on 3074 degrees of freedom
## Multiple R-squared:  0.4507, Adjusted R-squared:  0.4494 
## F-statistic: 360.3 on 7 and 3074 DF,  p-value: < 2.2e-16
## 
## [1] 145.1879
## [1] 145.1757
## [1] 0.4504273
## [1] 0.4506987
## [1] 0.449355
## [1] 0.4494479
\end{verbatim}

\begin{center}\rule{0.5\linewidth}{\linethickness}\end{center}

\subsection{Problem 4: Simulations in
R}\label{problem-4-simulations-in-r}

\subsubsection{a}\label{a-2}

Make R code that shows the interpretation of a 95\% CI for \(\beta_j\).
Hint: Theoretical question a.

\subsubsection{b}\label{b-2}

Make R code that shows the interpretation of a 95\% PI for a new
response at \({\bf x}_0\). Hint: Theoretical question b.

\subsubsection{c.}\label{c.}

For simple linear regression, simulate at data set with homoscedastic
errors and with heteroscedastic errors. Here is a suggestion of one
solution - not using \texttt{ggplot}. You use \texttt{ggplot}. Why this?
To see how things looks when the model is correct and wrong.

\begin{Shaded}
\begin{Highlighting}[]
\CommentTok{# Homoscedastic errore}
\NormalTok{n =}\StringTok{ }\DecValTok{1000}
\NormalTok{x =}\StringTok{ }\KeywordTok{seq}\NormalTok{(}\OperatorTok{-}\DecValTok{3}\NormalTok{, }\DecValTok{3}\NormalTok{, }\DataTypeTok{length =}\NormalTok{ n)}
\NormalTok{beta0 =}\StringTok{ }\DecValTok{-1}
\NormalTok{beta1 =}\StringTok{ }\DecValTok{2}
\NormalTok{xbeta =}\StringTok{ }\NormalTok{beta0 }\OperatorTok{+}\StringTok{ }\NormalTok{beta1 }\OperatorTok{*}\StringTok{ }\NormalTok{x}
\NormalTok{sigma =}\StringTok{ }\DecValTok{1}
\NormalTok{e1 =}\StringTok{ }\KeywordTok{rnorm}\NormalTok{(n, }\DataTypeTok{mean =} \DecValTok{0}\NormalTok{, }\DataTypeTok{sd =}\NormalTok{ sigma)}
\NormalTok{y1 =}\StringTok{ }\NormalTok{xbeta }\OperatorTok{+}\StringTok{ }\NormalTok{e1}
\NormalTok{ehat1 =}\StringTok{ }\KeywordTok{residuals}\NormalTok{(}\KeywordTok{lm}\NormalTok{(y1 }\OperatorTok{~}\StringTok{ }\NormalTok{x))}
\KeywordTok{plot}\NormalTok{(x, y1, }\DataTypeTok{pch =} \DecValTok{20}\NormalTok{)}
\KeywordTok{abline}\NormalTok{(beta0, beta1, }\DataTypeTok{col =} \DecValTok{1}\NormalTok{)}
\KeywordTok{plot}\NormalTok{(x, e1, }\DataTypeTok{pch =} \DecValTok{20}\NormalTok{)}
\KeywordTok{abline}\NormalTok{(}\DataTypeTok{h =} \DecValTok{0}\NormalTok{, }\DataTypeTok{col =} \DecValTok{2}\NormalTok{)}
\CommentTok{# Heteroscedastic errors}
\NormalTok{sigma =}\StringTok{ }\NormalTok{(}\FloatTok{0.1} \OperatorTok{+}\StringTok{ }\FloatTok{0.3} \OperatorTok{*}\StringTok{ }\NormalTok{(x }\OperatorTok{+}\StringTok{ }\DecValTok{3}\NormalTok{))}\OperatorTok{^}\DecValTok{2}
\NormalTok{e2 =}\StringTok{ }\KeywordTok{rnorm}\NormalTok{(n, }\DecValTok{0}\NormalTok{, }\DataTypeTok{sd =}\NormalTok{ sigma)}
\NormalTok{y2 =}\StringTok{ }\NormalTok{xbeta }\OperatorTok{+}\StringTok{ }\NormalTok{e2}
\NormalTok{ehat2 =}\StringTok{ }\KeywordTok{residuals}\NormalTok{(}\KeywordTok{lm}\NormalTok{(y2 }\OperatorTok{~}\StringTok{ }\NormalTok{x))}
\KeywordTok{plot}\NormalTok{(x, y2, }\DataTypeTok{pch =} \DecValTok{20}\NormalTok{)}
\KeywordTok{abline}\NormalTok{(beta0, beta1, }\DataTypeTok{col =} \DecValTok{2}\NormalTok{)}
\KeywordTok{plot}\NormalTok{(x, e2, }\DataTypeTok{pch =} \DecValTok{20}\NormalTok{)}
\KeywordTok{abline}\NormalTok{(}\DataTypeTok{h =} \DecValTok{0}\NormalTok{, }\DataTypeTok{col =} \DecValTok{2}\NormalTok{)}
\end{Highlighting}
\end{Shaded}

\begin{center}\rule{0.5\linewidth}{\linethickness}\end{center}

\subsection{Problem 5}\label{problem-5}

A problem with an interaction term between two continuous variables, and
between a continuous and a factor covariable with more than two levels.

Important: In the latter case, we need again the \(F\)-test to check if
the interaction is relevant.


\end{document}
