
Activities week 4 (January 21 and 24): Module 3 on Linear regression

Dear all!

1) Chill-week for the students who have done TMA4267 previously! There will not be any new stuff for you this week, see you on Monday January 28 for the Module 4: Classification! Of, cause you may do the Kahoot! that we will do on Thursday (link to come) - just to make sure that you are on top of things. Recommended exercises Problem 1 is very similar to the regression problem in Compulsory exercise 1 in 2019!

2) Plenary lectures this week on Monday January 21 at 8.15-10 in S4 (yes, I will bring the coffee maker), and Thursday January 24 in F6:
3LinReg.html 3LinReg.pdf
For this year TMA4267-students: you should come - this will give you an overview of the regression part of the TMA4267 course, and you will spend much time in February proving what we learn in TMA4268!

If it is some time since you did your first statistics course you may want to brush up on by browsing through some thematic pages and videos:

* What is the difference between a population and a random sample? 
Mette explains: https://mediasite.ntnu.no/Mediasite/Play/781f55191f7247f0aab97dec7cfec2c71d?catalog=8692a7e0-d315-45c7-84cb-384abb7b3d36
* What is a parameter and how do we do parameter estimation?
Håkon Tjelmeland explains: https://mediasite.ntnu.no/Mediasite/Play/c068ccc31d5442a08b451fb372c620851d?catalog=8692a7e0-d315-45c7-84cb-384abb7b3d36
Mette explains about maximum likelihood estimation:
https://mediasite.ntnu.no/Mediasite/Play/db9c6fbc45bf48abb8a4dd00ff146e081d?catalog=0fce6173-7a98-4db7-84b7-50cba3a3a341
Thematic pages: https://wiki.math.ntnu.no/tma4245/tema/begreper/estimator
* What is a confidence interval? What is a prediction interval?
Håkon Tjelmeland explains: https://mediasite.ntnu.no/Mediasite/Play/56bcb8e8bea74c0aa9ff7eff5287b1331d?catalog=8692a7e0-d315-45c7-84cb-384abb7b3d36
Thematic pages: https://wiki.math.ntnu.no/tma4245/tema/begreper/confidenceinterval
* How do we formulate and test an hypothesis and what is a p-value?
Mette explains about hypothesis testing: https://mediasite.ntnu.no/Mediasite/Play/cfdff7758b644d15b317027da29424901d?catalog=8692a7e0-d315-45c7-84cb-384abb7b3d36
Mette explains about P-value: https://mediasite.ntnu.no/Mediasite/Play/3d437740a70f41559c875573ff400d671d?catalog=0fce6173-7a98-4db7-84b7-50cba3a3a341
Thematic pages: https://wiki.math.ntnu.no/tma4245/tema/begreper/hypothesis
* What is the thinking behind simple linear regression?
Håkon explains: https://mediasite.ntnu.no/Mediasite/Play/773592a7c3a04a5295abdf01fa759b151d?catalog=8692a7e0-d315-45c7-84cb-384abb7b3d36
Mette works with betahat: https://mediasite.ntnu.no/Mediasite/Play/2e9a209c58874e75bd47e3c5e0b7b4e81d?catalog=0fce6173-7a98-4db7-84b7-50cba3a3a341
Thematic pages: https://wiki.math.ntnu.no/tma4245/tema/begreper/regression

3) Exercises: as usual in Smia on Thursday 24 at 16.15-18.

4) Did you find the M2 part on random vectors and the multivariate normal a bit too much? Then please join Mette in 734, 7. etg, sentralbygg 2 at Monday 21 January at 11.15 for a plenary lesson into these topics (I announced this for 10.15, but I have to attend a PhD trial lecture at 10.15-11, so this is rescheduled to 11.15, sorry that I forgot that - but maybe good for you for a break). 

Want to prepare? Here are some resources (which I will base my presentation on):

* Random vector: https://www.math.ntnu.no/emner/TMA4267/2017v/L1classnotes.pdf
* Moments: https://www.math.ntnu.no/emner/TMA4267/2017v/L2classnotes20170113.pdf

If you do not remember so well matrix algebra (Mathematics 3 at NTNU) you might want to brush up on that too:
[Härdle and Simes (2015): Applied Multivariate Statiistical Analysis, Chapter on "A short excursion into matrix algebra": https://link.springer.com/chapter/10.1007/978-3-540-72244-1_2 (on the reading list for TMA4267 Linear statistical models). NB: need to be logged in from NTNU (e.g. using vpn if you are not at NTNU) to see the ebook.

5) First meeting with the reference group is Thursday January 24 at 17.00 on table 8 in Smia: read agenda from message from last week. 

Have a great weekend - and then a great week!

-Mette
